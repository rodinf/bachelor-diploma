%!TEX TS-program = xelatex

% Author: Amet Umerov (admin@amet13.name)
% https://github.com/Amet13/bachelor-diploma
% Tuned for MAI by Rodin Fedor

% НАСТРОЙКИ ФОРМАТИРОВАНИЯ

\def \EmptyPageAfterTitle {}			% закомментировать, если не нужна пустая страница после титульной
\def \TwoSided {}						% если не закомментировано, то поля слева и справа будут чередоваться, нужно для двухсторонней печати. Если предполагается односторонняя печать, закомментируйте.

% НАСТРОЙКИ ИМЕН, УЧЕНЫХ СТЕПЕНЕЙ И Т.П.

\newcommand{\StudentFioLastname}{Родин}					% фамилия студента
\newcommand{\StudentFioFirstname}{Федор}				% имя студента
\newcommand{\StudentFioSurname}{Михайлович}				% отчество студента

\newcommand{\StudentKurs}{1}							% курс обучения студента
\newcommand{\StudentGroup}{М3О-121Б-20}					% группа студента

\newcommand{\PrepodFioLastname}{Гридин}					% фамилия преподавателя
\newcommand{\PrepodFioFirstname}{Александр}				% имя преподавателя
\newcommand{\PrepodFioSurname}{Николаевич}				% отчество преподавателя

\newcommand{\PrepodCaptionFirst}{преподаватель, }		% первая строчка ученой степени преподавателя
\newcommand{\PrepodCaptionSecond}{доцент, к. т. н.}		% вторая строчка ученой степени преподавателя


% НАСТРОЙКИ НАЗВАНИЙ РАБОТЫ

% тип работы (отчет, курсовая работа и т.д.)
\newcommand{\RabotaType}{ОТЧЕТ}

% о чем? (строчка после типа работы)
\newcommand{\RabotaOChem}{о выполнении задания по предмету «Дискретная математика»}

% тема или название работы
\newcommand{\RabotaThemeName}{«Новые учебно-­методические пособия по дискретной математике и онлайн-­калькуляторы диофантовых уравнений»}


% НАСТРОЙКИ ИНСТИТУТА И КАФЕДРЫ

% номер института
\newcommand{\InstitutNumber}{3}

% название института без кавычек
\newcommand{\InstitutName}{Системы управления, информатика и электроэнергетика}

% номер кафедры
\newcommand{\KafedraNumber}{307}

% название кафедры без кавычек
\newcommand{\KafedraName}{Цифровые технологии и информационные системы}


\ifx \TwoSided \undefined
	\documentclass[a4paper,titlepage,14pt]{extarticle}
\else
	\documentclass[a4paper,twoside,titlepage,14pt]{extarticle}
\fi

\input{inc/preamble} % Подключаем преамбулу
\usepackage{tabularx}
\usepackage{makecell}
%%% Начало документа
\begin{document}

%\includepdf{pz} % Пояснительная записка
%\includepdf[pages={1,2}]{task} % Задание на диплом печатается на одном листе с двух сторон
%помимо ПЗ и задания, в диплом также вкладывается отзыв руководителя и рецензия

% НАЧАЛО ТИТУЛЬНОГО ЛИСТА
\begin{titlepage}
\begin{center}
	
	\linespread{1.5}
	
	\normalsize{Министерство образования и науки Российской Федерации}\\
	\vspace{0.25cm}
	\normalsize{Федеральное государственное бюджетное образовательное\\ учреждение высшего образования}\\
	\vspace{0.25cm}
	\normalsize\textbf{«МОСКОВСКИЙ АВИАЦИОННЫЙ ИНСТИТУТ»}\\ {(НАЦИОНАЛЬНЫЙ ИССЛЕДОВАТЕЛЬСКИЙ УНИВЕРСИТЕТ)}\\
	\noindent\rule{\textwidth}{0.4pt} \\ \vspace{0.25cm}
	\normalsize
	{Институт № \InstitutNumber \ «\InstitutName»\\ Кафедра \KafedraNumber \ «\KafedraName»}\\
	\vfill
	
	{\RabotaType}\\
	{\RabotaOChem}\\
	\hfill\break	
	{\RabotaThemeName}\\
	
	\vfill
	
	\begin{tabularx}{0.95\textwidth}{ l X l l }
		
		Выполнил & \makecell[c]{\underline{\hspace{3cm}}} & \makecell[l]{студент \StudentKurs \ курса \\ группа \StudentGroup} & \makecell[r]{\StudentFioLastname \ \StudentFioFirstname \\ \StudentFioSurname} \\
		
		\multicolumn{4}{ c }{ } \\
		
		Проверил & \makecell[c]{\underline{\hspace{3cm}}} & \makecell[l]{\PrepodCaptionFirst \\ \PrepodCaptionSecond} & \makecell[r]{\PrepodFioLastname \ \PrepodFioFirstname \\ \PrepodFioSurname} \\
		
	\end{tabularx}

\end{center}
\hfill \break
\begin{center} Москва \the\year{} \end{center}
\thispagestyle{empty} % выключаем отображение номера для этой страницы
\end{titlepage}

\ifx \EmptyPageAfterTitle \undefined
\else
	\newpage
	\thispagestyle{empty}
	\mbox{}
	\newpage
\fi

%\newpagel

% КОНЕЦ ТИТУЛЬНОГО ЛИСТА

\tableofcontents % Содержание 
\clearpage

\input{inc/0-intro} % Введение
\input{inc/1-pz} % Постановка задачи
\input{inc/2-literature} % Обзор современных методов и технологий серверной виртуализации
\input{inc/3-analyze} % Системный анализ виртуальной инфраструктуры
\input{inc/4-description} % Описание виртуальной инфраструктуры
\input{inc/5-adminguide} % Руководство администратора
\input{inc/6-userguide} % Руководство пользователя
\input{inc/7-testing} % Результаты тестирования
\input{inc/8-lifesafety} % Безопасность жизнедеятельности
\input{inc/0-conclusion} % Заключение
\input{inc/0-bibliography} % Библиографический список

% Приложения
\input{inc/a-app} % Исходный код скрипта DDoS Deflate
\input{inc/b-app} % Руководство пользователя

\includepdf{act} % Акт внедрения

\end{document}
%%% Конец документа
